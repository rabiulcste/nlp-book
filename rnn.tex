\documentclass{article}
% adding fontspec pkg and kalpurush font
\usepackage{fontspec}
\usepackage{polyglossia}
\setdefaultlanguage{bengali}
\setotherlanguages{english}
\newfontfamily{\bengalifont}{Kalpurush}

\title{বাংলায় ম্যাশিন লার্নিং }
\author{রবিউল আউয়াল}
\date{২৬ জুন, ২০২১}


\begin{document}
\maketitle
\section{পরিচিত }
বাংলায় ম্যাশিং লার্নিংয়ে  বই লিখার চেষ্টা করছি। আজকে প্রথম কয়েকটি লাইন লিখলাম। ল্যাটেক্স কনফিগার করতে বেশ পেইন খেতে হয়েছে। এখন আরাম লাগছে।
\section{নিউরাল ল্যাঙ্গুয়েজ মডেল}
\subsection{রিকারেন্ট নিউরাল নেটস}
\subsection{ব্যাকপ্রোপাগেশন থ্রু টাইম (বিপিটিটি) }
\subsection{ভ্যানিশিং গ্র্যাডিয়েন্ট}
\subsection{লং শর্ট টার্ম মেমোরি নেটওয়ার্ক}

\end{document}