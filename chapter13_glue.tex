\chapter{GLUE বেঞ্চমার্ক টাস্কস}
গ্লু বেঞ্চমার্ক \cite{wang-etal-2018-glue}

\section{সেন্টেন্স ক্লাসিফিকেশন}
\section{ন্যাচারাল ল্যাঙ্গুয়েজ  ইনফারেন্স}
\section{কোশ্চেন আনসারিং}

আজকে আমরা কোশ্চেন আনসারিং নিয়ে আলাপ করবো। ইন্টারনেটে আমরা কিছু কুশ্চেন করতে পারি যেমনঃ

\begin{itemize}
	\item কখন প্রথম পিরামিড নির্মাণ করা হয়? 
	\item সাকিব আল হাসান
	\item জয়া আহসানের বয়স কত? 
	\item লন্ডন শহরের তাপমাত্রা কেমন?
	\item কেন আমরা ক্লান্ত লাগলে ঝিমাই?
	\item ক্রিস্টোফার নোলানের টপ ৫ টা মুভি 
\end{itemize}

কোশ্চেন অনেক রকমের হতে পারে এবং উত্তর বিভিন্ন ভাবে আসতে পারে। কোশ্চেন আনসার ডোমেইন কিভাবে কাজ করে সেটা আমরা দেখবো। 

\begin{table}[!ht]
\begin{tabular}{  p{15em}  p{10cm}|} 
  \toprule
  প্রশ্ন & উত্তর \\
  \bottomrule
  কখন প্রথম পিরামিড নির্মাণ করা হয়?  & ২৬৩০ BC \\ 
  \hline
  সাকিব আল হাসান & সাকিব আল হাসান একজন বাংলাদেশ ক্রিকেটার এবং অলরাউন্ডার। ২০০৬ সাল থেকে সাকিব বাংলাদেশ ক্রিকেট দলে খেলছেন। \\ 
  \hline
  জয়া আহসানের বয়স কত?  & ৪০ বছর  \\ 
  \hline
  লন্ডন শহরের তাপমাত্রা কেমন? & ৭ ডিগ্রী সেলসিয়াস। আকাশ পরিষ্কার, কিছুটা মেঘ দেখা যাচ্ছে।  \\ 
  \hline
  কেন আমরা ক্লান্ত লাগলে ঝিমাই? & বিশ্রামের জন্য মন কাঁদে।  \\ 
  \bottomrule
\end{tabular}
\caption{প্রশ্ন উত্তর উদাহরণ।}
\end{table}


কেন আমরা কোশ্চেন আনসারিং করবো? অনলাইনে বিলিওনস ডকুমেন্ট আছে। এআই এর মৌলিক একটা গোল হলো সার্চিং।  যেমন আপনি একটা ফ্লাইট বুক করতে চান। সেক্ষেত্রে আপনি চ্যাটবট ব্যবহার করে কোন হিউম্যান ইন্টারভেনশন ছাড়া ফ্লাইট বুক করতে পারছেন। যখন আমরা জিজ্ঞেস করছি জয়া আহসানের বয়স কতো; তখন সিস্টেমকে বিলিয়নস উইকিপিডিয়া ডকুমেন্ট থেকে সার্চ করে উত্তর খুঁজে বের করতে হচ্ছে; যাকে বলার ইনফরমেশন একট্র্যাকশন। ক্রিস্টোফার নোলানের টপ ৫ টি মুভির নাম এবং মূল থিম। এটা হলো সামারাইজেশন টাস্ক। QA সিস্টেম প্রথমে ডকুমেন্টগুলি ওয়েব সার্চিং টুল দিয়ে র‍্যাংক করবে রিলেভেন্ট ডকুমেন্টগুলি। সেগুলি হতে পারে আইএমডিবি বা রটেন টমেটো কিছু ওয়েব পেইজ। তারপর সেগুলি থেকে ১০০০ হাজার শব্দকে সামারাইজ করে হয়ত ১০০ শব্দের একটা সামারি তইরি করবে। লক্ষ্য করুন এখানে কোশ্চেন আনসারিং করতে ইনফরমেশন এক্ট্র্যাকশন এবং সামারাইজেশন দুইটা অতিরিক্ত কাজ করা লাগেব। কোশ্চেন আনসারিং সমস্যাকে বলা হয় এআই-কমপ্লিট। এর মানে হলো কোশ্চেন আনসারিং সমাধান করে ফেললে আমরা এআই এর অন্যান্য সমস্যাগুলিও সমাধান করে ফেলেছি। এছাড়া কোশ্চেন আনসারিং অনেকগুলি এপ্লিকেশন আছে। যেমন সার্চিং, ডায়ালগ, ইনফরমেশন একট্র্যাকশন, সামারাইজেশন। এই প্রত্যেকটা টুলই আমরা গুগোল সার্চ বা অন্যান্য ওয়েব সার্ভিব ব্যবহার করার সময় কাজে লাগাচ্ছি  যার মূল ভিত্তি কোশ্চেন আনসারিং। প্রথম সফল কোশ্চেন আনসারিং সিস্টেম ধরা হয় আইবিএম জিওপার্ডিকে। যদিও পরে যতোটা ঢাকঢোল পিটানো হয়েছিল আইবিএম সেসব সত্যি কাজ করে নি। আমার দৈনন্দিন জীবনে সার্চিং, ফ্লাইট রিজার্ভেশন, পণ্য অর্ডার করা, সিরি এগুলি নিয়মিত সফল QA সিস্টেম। এছাড়াও আরো বিভিন্ন রকমের সমস্যা আছে। যেমন একটা প্রশ্ন হতে পারেঃ সাকিব আল হাসান কতগুলি টি টুয়েন্টি ম্যাচে ৫০ করেছেন। এই কাজের জন্য ম্যাশিনকে একটা টেব্যুলার ড্যাটা থেকে কাউন্টিং করে উত্তর খুঁজে বের করতে। তাহলে আমরা দেখতে পাচ্ছি QA সিস্টেম গাণিতিক সমস্যা সমাধান করার প্রয়োজন পড়ে। আরেকটি কঠিন সমস্যা হলো রিজনিং। যেসব প্রশ্নের উত্তর ডিরেক্টলি দেয়া সম্ভব না। যেগুলি উত্তর করতে গেলে রিজনিং করে উত্তর দিতে হয়। যেমনঃ যেকোন ধরণের মোরাল এথিক্যাল কুশ্চেন। কেউ যদি সিরিকে প্রশ্নকে আমার মা আমাকে বকা দিয়েছে, আমি কি নিজেকে কষ্ট দিব ওর উপর রাগ করে? এই ধরণের প্রশ্নের উত্তর দিতে গেলে এআই সিস্টেমকে হিউম্যান ইন্টেলিজেন্সের আস্পেক্টগুলি যেমন মোরালিটি, এথিকস, রিজনিং, মিনিং এগুলি বুঝতে হবে। QA আমরা নিত্যকার জীবনে রোজ ব্যবহার করছি।  সিরি প্রশ্ন উত্তর, গুগোল সার্চের একটা বড় অংশ QA সিস্টেম। 


\begin{table}[!ht]
\begin{tabular}{  p{15em}  p{10cm}} 
  \toprule
  প্রশ্ন & উত্তরের উৎস \\
  \bottomrule
  কখন প্রথম পিরামিড নির্মাণ করা হয়?  & এনসাইক্লোপিডিয়া \\ 
  \hline
  সাকিব আল হাসান & সাম্প্রতিক এনসাইক্লোপিডিয়া / উইকিপিডিয়া \\ 
  \hline
  জয়া আহসানের বয়স কত?  &  পত্রিকা / উইকিপিডিয়া \\ 
  \hline
  লন্ডন শহরের তাপমাত্রা কেমন? & আবহাওয়া দফতর  \\ 
  \hline
  কেন আমি পরীক্ষায় ফেল করেছি & ব্যক্তিগত অবজার্ভেশন \\  
  কেন আমরা ক্লান্ত লাগলে ঝিমাই? &  বিভিন্ন গবেষণাপত্র  \\ 
  \bottomrule
\end{tabular}
\caption{প্রশ্ন এবং উত্তরের বিভিন্ন উৎস।}
\end{table}

প্রথম দুটি উত্তর পাবো উইকিপিডিয়া থেকে। স্টক মার্কেটের তথ্য স্টক একচেঞ্জ ড্যাটাবেজ থেকে। আবহাওয়ার খবর আবহাওয়া দফতর থেকে। নিজের ব্যক্তিগত অব্জার্ভেশন থেকে উত্তর পাচ্ছি। কিছু ক্ষেত্রে বিভিন্ন গবেষণা পত্র পড়ে উত্তর খুঁজে নিতে পারি। কোশ্চেন আনসারিং সিস্টেম ৩ রকমের ড্যাটার উপর নির্ভর করে। 



\begin{table}[!ht]
\begin{tabular}{  p{10em}  p{5cm} p{7cm}} 
  \toprule
  প্রশ্ন & কন্টেক্সট / উৎস & উত্তর \\
  \bottomrule
  ফ্যাকচুয়াল প্রশ্ন  & একগুচ্ছ ডকুমেন্ট (করপাস) & একটাই ফ্যাক্ট  \\ 
  জটিল / বর্ণনামূলক প্রশ্ন  & & একটা এক্সপ্ল্যানাশন \\ 
  ইনফরমেশন রিট্রাইভাল  &  একটি ডকুমেন্ট &  \\ 
  & & নলেজ বেইজ   \\ 
  & & বাক্য কিংবা প্যারাগ্রাফ কোথাও থেকে এক্সট্র্যাক্ট করা \\  
  & নন টেক্সুয়াল ড্যাটা (ছবি, সেন্সর, জিপিএস)  & একটি ছবি বা অন্যান্য অবজেক্ট \\
   & & আরেকটি প্রশ্ন \\ 
  \bottomrule
\end{tabular}
\caption{প্রশ্ন এবং উত্তরের বিভিন্ন উৎস।}
\end{table}

এই রকমফেরের উপর নির্ভর করে আমরা বিভিন্ন রকম QA সিস্টেম ডিজাইন করার কথা ভাবতে পারি। আমরা দেখতে পাচ্ছি প্রশ্ন ফ্যাকচুয়াল (অনেকগুলি ডকুমেন্ট খুঁজে একটাই উত্তর; হ্যাঁ/ না, বছর) , বর্ণনামূলক  কুশ্চেন (একটা করপাস ঘেটে একটা ব্যখ্যা তইরি) এবং ইনফরমেশন রিট্রাইভাল (একটা ডকুমেন্ট,  কিংবা নলেজ বেইজ থেকে বা ছবির ড্যাটাবেজ একটা ছবি বের করা)। কমপ্লেক্স বা ন্যারেটিভ  প্রশ্নের উদাহরন হতে পারে আমার কম্পিউটার ব্লু স্ক্রিন কেন হয়ে যায় মাঝে মাঝে এবং এটা কিভাবে ফিক্স করতে পারি? আরেকটি জনপ্রিয় উৎস হচ্ছে নলেজ বেইজ যা অনেক পুরান এনএলপি রিসার্চ টেকনিক। এটা এখনো অনেক একটিভ এবং কঠিন রিসার্চ এরিয়া। অনেক সময় প্রশ্নের উত্তরে পাল্টা প্রশ্ন করা হয় নির্দিষ্ট প্রশ্ন বা নিশ্চিত হবার জন্য। সাধারণত ডায়ালগ সিস্টেমে সেই ধরণের পাল্টা প্রশ্ন উত্তর ইন্টারফেইস থাকে। যেমন আমি যদি হোটেল বুকিং সিস্টেমে প্রশ্ন করি আগামী ৭ দিনে মন্ট্রিয়ল শহরে কোন হোটেল পাওয়া যাবে কিনা? তখন সিস্টেম প্রশ্ন করতে পারে আমার কোন পছন্দের এলাকা আছে কিনা? আমার উত্তরের সাপেক্ষে সার্চ আরো ন্যারো ডাউন করার জন্য সিস্টেম আমাকে জিজ্ঞেস করতে পারে আমার কয়টা রুম লাগবে বা কজন থাকবো? এটা হলো প্রশ্নের উত্তর পাল্টা প্রশ্ন যেগুলি নিয়মিত আমরা নিত্যদিন ব্যবহার করছি।

\subsubsection{প্রশ্নের ট্যাক্সোনোমি}
অনেক রকম ট্যাক্সোনোমি হতে পারে প্রশ্নের জন্যঃ 

\begin{itemize}
  \item কি কেন কখন শব্দ দিয়ে প্রশ্ন  (Wh- কুশ্চেন)
  \item প্রশ্নের বিষয় 
  \item উত্তরের আকার (form) 
  \item উৎসের টাইপ যেখান থেকে উত্তর পাওয়া যেতে পারে  
\end{itemize}

এই ট্যাক্সোনোমিগুলি মাথায় রেখে আমরা প্রশ্ন উত্তর সিস্টেম বানাতে পারি। যেমন আমাদের স্কুলে যে রিডিং কম্প্রিহেনশন থাকতো সেখানে উত্তর অই নির্দিষ্ট প্যাসেজ থেকেই খুঁজে বের করতো হতো? তাহলে আমরা যদি অমন একটা প্রশ্ন উত্তর সিস্টেম বানাতে চাই যেটা রিডিং কম্প্রিহেনশন করতে পারবে আমাদের ম্যাশিন লার্নিং টাস্কটা ওয়েল ডিফাইনড হবে। তাহলে কি ধরণের উৎস থেকে উত্তর পাচ্ছি সেটার উপর নির্ভর করে QA টাস্ক তইরি করা এবং ড্যাটাসেট এভালুয়েশন এগুলি সহজ হয়। কাজের সুবিদ্ধার্থে সাধারণত প্রশ্ন উত্তর সিস্টেম বানানোর জন্য উত্তরের উৎস চিন্তা করে আমরা কাজ শুরু করে দিতে পারি। কারণ ড্যাটা সোর্স ম্যাশিন লার্নিং ডিজাইনের জন্য সবচে ক্রিটিক্যাল উপাদান। বলা যেতে পারে, আমরা প্রশ্নের চে উত্তরের উপর বেশি ফোকাস করছি। 

এখন আমাদের কাছে প্রশ্ন উত্তর সিস্টেম পরের ধাপে ৩ টি প্রশ্ন মাথায় রাখতে হবেঃ 
\begin{itemize}
  \item উত্তর দেখতে কেমন হবে। যেমনঃ হ্যাঁ/না, ড্যাটাবেইজ লুক আপ,  প্যাসেজ থেকে একটা বাক্যাংশ ইত্যাদি।  
  \item আমরা উত্তর কোথায় পাবো (উত্তরের উৎস)? যেমনঃ উইকিপিডিয়া, রিডিং প্যাসেজ, ড্যাটাবেইজ টেবিল ইত্যাদি।  
  \item আমার ট্রেইনিং ড্যাটা দেখতে কেমন হবে (ম্যাশিন লার্নিং টাস্ক)? একটা প্যাসেজ থাকবে এবং প্যাসেজ থেকে কিছু রিলেভেন্ট প্রশ্ন। উত্তর হবে প্যাসেজের বাক্যাংশ।
\end{itemize}


আমরা এখন দেখবো লিটারেচারে কিভাবে কোশ্চেন আনসারিং টাস্ক কিভাবে ডিফাইন করা হয়েছে। নিচে কয়েকটি কোশ্চেন আনসারিং এরিয়া লিস্ট ডাউন করা হলোঃ 

\begin{table}[!ht]
  \begin{tabular}{ p{8em}  p{12cm}}
  \textbf{রিডিং কম্প্রিহেনশন} &  \tabitem উত্তর একটা ডকুমেন্ট থেকে পাবো \tabitem কন্টেক্সট হলো ডকুমেন্টটি \\  
  \midrule 
  \textbf{সিমান্টিক পার্সিং} & \tabitem উত্তর একটি লজিক্যাল ফরম, একটা নজেল বেইজে এক্সিকিউট করা যায় \tabitem কনটেক্সট হলো নলেজ বেইজ \\ 
   \midrule
  \textbf{ভিজুয়াল QA} & \tabitem  উত্তর সহজ এবং ফ্যাকচুয়াল \tabitem উত্তর এক বা একের অধিক ছবি \\
   \midrule 
  \textbf{ইনফরমেশন রিট্রাইভাল} & \tabitem উত্তর একটি ডকুমেন্ট / প্যারাগ্রাফ / বাক্য \tabitem কনটেক্সট একটি ডকুমেন্টের করপাস \\  
  \end{tabular}
\end{table}




রিডিং কম্প্রিহেনসন আমরা উচ্চ মাধ্যমিকে পড়েছি। একজ্যাক্টলি একই কাজ আমরা এখন ম্যাশিনকে দিয়ে করাবো। এবার ধরা যাক আমাদের ড্যাটাবেইজে একটা টেবিল আছে যেখানে প্রতি সারিতে বাংলাদেশ খেলোয়াড়রা কোন বছরে কতোগুলি ৫০ করেছেন তার তথ্য আছে। এটা একটা স্ট্রাকচারড ড্যাটা সোর্স। এখন আমরা আগে যে প্রশ্নটি করলাম সাকিব কতোগুলি ৫০ করেছেন? এই প্রশ্নটি একটা ন্যাচারল কোশ্চেন। আমরা এটা ডিরেক্টলি ড্যাটাবেইজ কুয়েরি করতে পারবো না। এটা উত্তর পেতে আমরা প্রথমে ন্যাচারাল ল্যাঙ্গুয়েজটাকে SQL এ রূপান্তর করবো এবং তারপর ড্যাটাবেইজ কুয়েরি করতে পারবো। এই টাস্কটিকে ম্যাশিন লার্নিংয়ে বলা হয় সিমান্টিক পার্সিং। এটা এন্টারপ্রাইজ সল্যুশন বিশেষত ব্যাংকিং এ প্রচুর ব্যবহৃত হয়।  ন্যাচারাল ল্যাঙ্গুয়েজ থেকে যেকোন ড্যাটাবেইজ কুয়েরি রূপান্তর করাই সিমান্টিক পার্সিং। ভিজুয়াল কুশ্চেন আনসারিংযে আমাদের একটা ভিশন সিস্টেম থাকবে এবং একটা ল্যাংগুয়েজ সিস্টেম থাকে। ভিশন সিস্টেম ইমেজ প্রসেস করে এবং ন্যাচারাল ল্যাংগুয়েজ সিস্টেম ইমেইজের মধ্যে যে তথ্য বা অবজেক্টগুলি সেগুলি নিয়ে প্রশ্ন করলে; সেটা উত্তর হিশেবে দেয়। যেমন নিচের ছবিতে একটা প্রশ্ন করতে পারি আমরা কয়টা হলুদ রঙের কলা আছে? অন্যান্য QA টাস্কের তুলনায় ভিজুয়াল QA তুলনামূলক বেশি কঠিন কেননা একই সাথে ভিশন এবং ল্যাগুয়েজ দুটি সমস্যা নিয়ে কাজ করতে হয়। ইনফরমেশন রিট্রাইভাল হলো ক্লাসিক ড্যাটা মাইনিং সমস্যা। আমাদের নেইম এন্টিটি একটা ইনফরমেশন রিট্রাইনভাল টাস্ক। ওয়েব ডকুমেন্ট সার্চিং আরেকটি উদাহরণ। 



\subsection{রিডিং কম্প্রিহেনশন}
\subsection{ওপেন ডোমেইন কুশ্চেন আনসারিং}






